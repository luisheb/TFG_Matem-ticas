\documentclass[a4paper,11pt,spanish, twoside, leqno]{tfg-uam}

\usepackage[utf8]{inputenc}
\usepackage{amsfonts, amssymb, amsmath, amsthm}
\usepackage{graphicx}
\usepackage{color}
\usepackage{xcolor}

\newtheorem{teor}{Teorema}[chapter]
\newtheorem{lema}[teor]{Lema}
\newtheorem*{teorsin}{Teorema}


\theoremstyle{definition}
\newtheorem{defin}[teor]{Definici\'on}

\title{Redes Neuronales: aproximación de EDPs}
\author{Luis Hebrero Garicano}
\tutor{Julia Novo}
\curso{2024-2025}


%%%%%METADATOS: rellenar la info solicitada entre llaves
\usepackage{hyperref}
\hypersetup{
	pdfinfo={
            Title={Redes Neuronales: aproximacion de EDPs}, %Titulo del trabajo; ejemplo: Matematicas y desarrollo
            Author={Luis Hebrero Garicano}, %Autor del trabajo; ejemplo: Juan Sanchez
            Director1={julia.novo}, %Tutor1: en formato nombre.apellido, tal como aparece en la primera parte, antes de la arroba,  de su direcci�n de correo electr�nico de la UAM; ejemplo: fernando.soria
            Director2={ }, %Tutor2: en formato nombre.apellido, tal como aparece en la primera parte, antes de la arroba,  de su direcci�n de correo electr�nico de la UAM
            Ndirectores={1}, %Numero total de directores: 1 � 2
            Tipo={TFG}, %no tocar
            Curso={2024-25}, %no tocar
            Palabrasclave={ },% Palabras clave del trabajo, separadas por comas y sin acentos ni espacios; ejemplo: morfismos, formas modulares, ecuaciones elipticas
				}
}
%%%%%%%%%%%%%%%%%%%%%%%%%%%%%%%

\begin{document}



\begin{abstract}[spanish]
Las redes neuronales son un modelo matemático inspirado en el funcionamiento cerebral que, en esencia, se utiliza para encontrar funciones: funciones que clasifican datos, que predicen valores o incluso que anticipan la siguiente palabra en una frase. En este trabajo, se estudiará cómo se puede aplicar esta capacidad de aproximación de funciones para resolver ecuaciones en derivadas parciales. Exploraremos distintas estrategias para construir estas redes y analizaremos su aplicación en problemas concretos, centrándonos en las ventajas que aportan con respecto a los métodos numéricos tradicionales, así como en los casos en los que una aproximación mediante redes neuronales no resulta efectiva.
\end{abstract}
\begin{abstract}[english]
Neural networks are a mathematical model inspired by the brain's functioning, primarily used to find functions: functions that classify data, predict values, or even anticipate the next word in a sentence. This work will explore how this function approximation capability can be applied to solve partial differential equations. We will investigate different strategies for constructing these networks and analyze their application to specific problems, focusing on the advantages they offer over traditional numerical methods, as well as the cases where a neural network-based approach proves ineffective.
\end{abstract}
\mainmatter


\chapter{Introducción y preliminares}\label{chap1}
\setcounter{page}{1}
Para poder entender el las aproximaciones a las EDPs mediante redes neuronales, es necesario tener un conocimiento previo de las redes neuronales y de las ecuaciones en derivadas parciales. En este capítulo, se introducirán los conceptos básicos de ambos temas, así como las herramientas matemáticas necesarias para comprender el resto del trabajo.

\section{Introducción a las redes neuronales}
Una red neuronal, de forma abstracta, es simplemente una función que toma una entrada y produce una salida. Es decir, una red neuronal, es una función $F$ que toma un vector de entrada $x$ y produce un vector de salida $y$, siendo $F: \mathbb{R}^n \rightarrow \mathbb{R}^m$. La red neuronal se compone de una serie de capas, cada una de las cuales está formada por un conjunto de neuronas. Cada neurona de una capa recibe una serie de entradas, las procesa y produce una salida. La salida de cada neurona se calcula mediante una función de activación, que puede ser de distintos tipos, como la función sigmoide, la función tangente hiperbólica o la función ReLU. Para entender este concepto nos vamos a centrar en el caso concreto de la red neuronal de la Figura \ref{fig:RedNeuronal}.

\begin{figure}[!ht]
    \centering
    \includegraphics[width=\textwidth]{RedNeuronal.png}
    \caption{Esquema de una red neuronal con 4 capas.}
    \label{fig:RedNeuronal}
\end{figure}

Como se ve en la Figura \ref{fig:RedNeuronal}, la entrada en nuestra función está representada por los dos círculos de la izquierda, que representan los valores de entrada $x_1$ y $x_2$, siendo así la entrada $x\in\mathbb{R}^2$. Estos valores se multiplican por unos pesos ($W^{[2]}$)y se suman a un sesgo $b$. La salida de esta neurona se calcula mediante una función de activación. Así, los valores que ``llegan'' a la segunda capa de nuestra red neuronal serán de la forma
\begin{equation*}
    \sigma(W^{[2]}x+b^{[2]})\in\mathbb{R}^2,
\end{equation*}

Siendo $W^{[2]}\in\mathbb{R}^{2\times2}$ y el vector $b^{[2]}\in\mathbb{R}^2$. A partir de aquí, se repite el proceso para cada capa de la red neuronal, hasta llegar a la capa de salida, que nos dará el valor de salida de nuestra red neuronal. De forma visual, se pueden interpretar las flechas de la Figura \ref{fig:RedNeuronal} como los pesos por los que se va multiplicando.

En la tercera capa de la red neuronal, vemos que los valores que llegan de la capa 2, estos $\sigma(W^{[2]}x+b^{[2]})$, pertenecen a $\mathbb{R}^2$. De este modo, como tenemos 3 neuronas en la tercera capa, para obtener un valor perteneciente a $\mathbb{R}^3$, necesitamos una matriz $W^{[3]}\in\mathbb{R}^{3\times2}$ y un vector $b^{[3]}\in\mathbb{R}^3$. Así, el valor de nuestra red neuronal en la tercera capa será
\begin{equation*}
    \sigma(W^{[3]}\sigma(W^{[2]}x+b^{[2]})+b^{[3]})\in\mathbb{R}^3.
\end{equation*}

Finalmente, en la capa de salida recivirá de la tercera capa un vector perteneciente a $\mathbb{R}^3$, por lo que necesitaremos una matriz $W^{[4]}\in\mathbb{R}^{3\times3}$ y un vector $b^{[4]}\in\mathbb{R}^3$. Así, el valor de salida de nuestra red neuronal, esa $F$ de la que habíamos hablado al principio, será
\begin{equation}
    F(x)=\sigma(W^{[4]}\sigma(W^{[3]}\sigma(W^{[2]}x+b^{[2]})+b^{[3]})+b^{[4]})\in\mathbb{R}^3.
\end{equation}\label{eq:RedNeuronal}

En general, una red neuronal se puede representar como una composición de funciones, donde cada función es una capa de la red neuronal. 

Nuestra intención con este tipo de funciones es ir variando los valores de las matrices $W$, también conocidos como pesos, para que la salida de nuestra red neuronal se acerque lo máximo posible a la salida deseada. 

Para entender esto, vamos a utilizar la red neuronal de la Figura \ref{fig:RedNeuronal} para resolver un problema concreto muy sencillo de clasificación. Supongamos que tenemos una serie de puntos en el plano, de tres tipos distintos, como los de la Figura \ref{fig:Clasificacion}, y queremos clasificarlos en tres grupos, los puntos de tipo azul, rojo y amarillo.


\begin{figure}[!ht]
    \centering
    \includegraphics[width=\textwidth]{Figuras/pic_xy.png}
    \caption{Puntos en el plano que marcan las dos categorías}
    \label{fig:Clasificacion}
\end{figure}

De este modo, nuestra red neuronal recivirá como entrada un punto del plano, y nos dirá a qué categoría pertenece. devolviendo $(1,0,0)^T$ si es de la categoría azul, $(0,1,0)^T$ si es de la categoría roja y $(0,0,1)^T$ si es de la categoría amarilla.

Lo siguiente que querremos hacer será entrenar la red neuronal, es decir, ajustar los pesos y sesgos de la red neuronal para que la salida se acerque lo máximo posible a la salida deseada. Es decir, que cuando se introduzca un punto de un tipo concreto, la salida lo asigne a la categoría adecuada. 

Designamos a $y(x)$ como la salida deseada de nuestra red neuronal, y a $F(x)$ como la salida real. Así, el error vendrá dado en función de los pesos y sesgos de la siguiente forma
\begin{equation*}
    E(W^{[2]},W^{[3]},W^{[4]},b^{[2]},b^{[3]},b^{[4]})=\frac{1}{2}\sum_{x\in X}\|y(x)-F(x)\|^2,
\end{equation*}

Donde $X$ es el conjunto de puntos que tenemos para entrenar la red neuronal, en nuestro caso, serán 15. Así, lo que querremos hacer es minimizar esta función de error, es decir, encontrar los pesos que minimicen la función $E$. Para ello, se utilizan algoritmos de optimización, como el descenso del gradiente. Este proceso es conocido como el entrenamiento de la red neuronal. Si se logra con éxito, la red neuronal será capaz de clasificar correctamente los puntos en el plano. En este caso concreto, al entrenar la red neuronal, se obtiene la clasificación de la Figura \ref{fig:ClasificacionFinal}, en la que simplemente hemos aplicado nuestra función para cada punto del plano y lo hemos sombreado de acuerdo a la clasificación que se le ha dado.

\begin{figure}[!ht]
    \centering
    \includegraphics[width=\textwidth]{Figuras/classifier_back.png}
    \caption{Puntos en el plano que marcan las dos categorías}
    \label{fig:ClasificacionFinal}
\end{figure}

Más adelante entenderemos más en detalle como es este proceso de entrenamiento.

\subsection{Comentarios sobre la función sigmoide}
Como el lector habŕa adelantado ya, las redes neuronales sirven úncamente para aproximar o encontrar funciones. No obstante, para que este objetivo tenga sentido, tenemos que poder aproximar todo tipo de funciones. De otra forma, igual podríamos estar entrenando una red neuronal para encontrar una función $F$ que por definición no puede encontrar.

Para poder lograr este objetivo, se necesitan las funciones de activación. Como ejercicio, supongamos que en nuestra ecuación \ref{eq:RedNeuronal} en lugar de utilizar la función sigmoide, no utilizamos ninguna función de activación, es decir, que nuestra red neuronal es simplemente una composición de funciones lineales. En este caso, nuestra red tendría la siguiente forma

\begin{equation*}
    F(x)=W^{[4]}(W^{[3]}(W^{[2]}x+b^{[2]})+b^{[3]})+b^{[4]}\in\mathbb{R}^3.
\end{equation*}

Tratándose así, claramente de una función lineal. Por tanto, una función lineal sería incapaz de aproximar una no lineal. Para poder aproximar cualquier función, se necesitan funciones de activación no lineales. La función sigmoide es una de las más utilizadas, pero existen otras como la función tangente hiperbólica o la función ReLU. En este trabajo, nos centraremos en la función sigmoide, que se define como
\begin{equation*}
    \sigma(x)=\frac{1}{1+e^{-x}}.
\end{equation*}

Esta función tiene la ventaja de que su derivada es fácil de calcular, y es precisamente esta derivada la que se utiliza en el algoritmo de entrenamiento de la red neuronal.

Si usamos funciones de activación no lineales, como la función sigmoide, podemos demostrar lo siguiente
\begin{teor}
    Sea $\sigma\in C(\mathbb{R})$. El conjunto de redes neuronales con $\sigma$ como funcion de activación, es denso en $C(\mathbb{R}^n)$ si y solo si $\sigma$ no es una función polinómica.
\end{teor}
\colorbox{yellow}{FALTA DEMOSTRACIÓN!!! !!!Como se referencian bibliografía!!! }

\colorbox{yellow}{Lo de espacio topológico!!!}


\section{Introducción a las ecuaciones en derivadas parciales}

Las ecuaciones en derivadas parciales (EDPs) son ecuaciones que relacionan una función desconocida con sus derivadas parciales. Son fundamentales en la física y en la ingeniería, ya que permiten modelar fenómenos físicos y predecir su evolución en el tiempo. Las EDPs se dividen en dos tipos: las ecuaciones en derivadas parciales elípticas, que modelan fenómenos estacionarios, como la distribución de temperatura en un sólido, y las ecuaciones en derivadas parciales parabólicas, que modelan fenómenos que evolucionan en el tiempo, como la difusión de una sustancia en un fluido. En este trabajo, nos centraremos en las ecuaciones en derivadas parciales elípticas, que son las más sencillas de resolver mediante redes neuronales.
\section{Métodos numéricos tradicionales: el método de los elementos finitos}
El método de los elementos finitos es uno de los métodos numéricos más utilizados para resolver ecuaciones en derivadas parciales. Consiste en dividir el dominio de la ecuación en un conjunto de elementos finitos, y aproximar la solución de la ecuación en cada elemento mediante una función polinómica. La solución de la ecuación se obtiene entonces como una combinación lineal de las soluciones en cada elemento. El método de los elementos finitos es muy eficaz para resolver ecuaciones en derivadas parciales elípticas, pero presenta algunas limitaciones, como la necesidad de discretizar el dominio de la ecuación y la dificultad de tratar con geometrías complejas. En este trabajo, exploraremos cómo las redes neuronales pueden superar estas limitaciones y proporcionar una alternativa eficaz y flexible para resolver ecuaciones en derivadas parciales.
\colorbox{yellow}{Todo lo que sea necesario para poder entender el resto del trabajo.}

\chapter{Aproximación de EDPs mediante redes neuronales}\label{chap2}

\section{PINS }
\section{El ``Deep Ritz Method''}

\chapter{Resultados}\label{chap3}





\headrule
\newpage
\section{Notas del formato}
\begin{teorsin}
[Cauchy--Schwarz]Nullam quis ante. 
\end{teorsin}

\begin{teor}\label{teor1}
Nullam quis ante. Etiam sit amet orci eget eros faucibus tincidunt. Duis leo. Sed fringilla mauris sit amet nibh. Donec sodales sagittis magna. Sed consequat, leo eget bibendum sodales, augue velit cursus nunc.
\end{teor}

\begin{defin}
Nullam quis ante. Etiam sit amet orci eget eros faucibus tincidunt. Duis leo. Sed fringilla mauris sit amet nibh. Donec sodales sagittis magna. Sed consequat, leo eget bibendum sodales, augue velit cursus nunc.
\end{defin}
\begin{lema}\label{lema1}
Nullam quis ante. Etiam sit amet orci eget eros faucibus tincidunt. Duis leo. Sed fringilla mauris sit amet nibh. Donec sodales sagittis magna. Sed consequat, leo eget bibendum sodales, augue velit cursus nunc.
\end{lema}
\begin{proof}
Nullam quis ante. 
\end{proof}


\begin{lema}\label{lema2}
Nullam quis ante. Etiam sit amet orci eget eros faucibus tincidunt. Duis leo. Sed fringilla mauris sit amet nibh. Donec sodales sagittis magna. Sed consequat, leo eget bibendum sodales, augue velit cursus nunc.
\end{lema}

Lorem ipsum dolor sit amet, consectetuer adipiscing elit. Aenean commodo ligula eget dolor. Aenean massa. Cum sociis natoque penatibus et magnis dis parturient montes, nascetur ridiculus mus. Donec quam felis, ultricies nec, pellentesque eu, pretium quis, sem. Nulla consequat massa quis enim. Donec pede justo, fringilla vel, aliquet nec, vulputate eget, arcu. In enim justo, rhoncus ut, imperdiet a, venenatis vitae, justo. Nullam dictum felis eu pede mollis pretium. Integer tincidunt. Cras dapibus. Vivamus elementum semper nisi. Aenean vulputate eleifend tellus. Aenean leo ligula, porttitor eu, consequat vitae, eleifend ac, enim. Aliquam lorem ante, dapibus in, viverra quis, feugiat a, tellus. Phasellus viverra nulla ut metus varius laoreet. Quisque rutrum. Aenean imperdiet. Etiam ultricies nisi vel augue. Curabitur ullamcorper ultricies nisi. Nam eget dui. Etiam rhoncus. Maecenas tempus, tellus eget condimentum rhoncus, sem quam semper libero, sit amet adipiscing sem neque sed ipsum. Nam quam nunc, blandit vel, luctus pulvinar, hendrerit id, lorem. Maecenas nec odio et ante tincidunt tempus. Donec vitae sapien ut libero venenatis faucibus. Nullam quis ante. Etiam sit amet orci eget eros faucibus tincidunt. Duis leo. Sed fringilla mauris sit amet nibh. Donec sodales sagittis magna. Sed consequat, leo eget bibendum sodales, augue velit cursus nunc.


\begin{proof}[\sc Demostraci\'on del lema {\rm \ref{lema2}}]
Nullam quis ante:
\begin{equation}
2+2=4.\qedhere
\end{equation}
\end{proof}

Lorem ipsum dolor sit amet, Teorema \ref{teor1}, consectetuer adipiscing elit. Aenean commodo ligula eget dolor. Aenean massa. Cum sociis natoque penatibus et magnis dis parturient montes, nascetur ridiculus mus. Donec quam felis, ultricies nec, pellentesque eu, pretium quis, sem. Nulla consequat massa quis enim. Donec pede justo, fringilla vel, aliquet nec, vulputate eget, arcu. In enim justo, rhoncus ut, imperdiet a, venenatis vitae, justo. Nullam dictum felis eu pede mollis pretium. Integer tincidunt. Cras dapibus. Vivamus elementum semper nisi. Aenean vulputate eleifend tellus. Aenean leo ligula, porttitor eu, consequat vitae, eleifend ac, enim. Aliquam lorem ante, dapibus in, viverra quis, feugiat a, tellus. Phasellus viverra nulla ut metus varius laoreet. Quisque rutrum. Aenean imperdiet. Etiam ultricies nisi vel augue. Curabitur ullamcorper ultricies nisi. Nam eget dui. Etiam rhoncus. Maecenas tempus, tellus eget condimentum rhoncus, sem quam semper libero, sit amet adipiscing sem neque sed ipsum. Nam quam nunc, blandit vel, luctus pulvinar, hendrerit id, lorem. Maecenas nec odio et ante tincidunt tempus. Donec vitae sapien ut libero venenatis faucibus. Nullam quis ante. Etiam sit amet orci eget eros faucibus tincidunt. Duis leo. Sed fringilla mauris sit amet nibh. Donec sodales sagittis magna. Sed consequat, leo eget bibendum sodales, augue velit cursus nunc,
\begin{align}\label{eq4}
&e^{i\pi }+1=0,
\\
&2e^{i\pi }+2=0.\label{eq5}
\end{align}

Lorem ipsum dolor sit amet, consectetuer adipiscing elit. Aenean commodo ligula eget dolor.
\begin{align}\nonumber
0&=e^{i\pi }+1=e^{i\pi }+1=e^{i\pi }+1=e^{i\pi }+\sum_{n=1}^\infty \frac{1}{2^n}
\\
&=-1+\sum_{n=1}^\infty \frac{1}{2^n}=-1+1=0.\label{eq6}
\end{align}
et
\begin{equation}\label{eq7}
\begin{aligned}
e^{i\pi }+1=0,
\\
e^{i\pi }+1=0.
\end{aligned}
\end{equation}


Lorem ipsum dolor sit amet, consectetuer adipiscing elit. Aenean commodo ligula eget dolor.
\begin{align*}
e^{i\pi }+1&=0,
\\
e^{i\pi }+1&=0.
\end{align*}
Aenean massa: 
\begin{equation}
\left\{
\begin{array}{l}
e^{i\pi }+1=0,
\\
e^{i\pi }+1=0.
\end{array}
\right.
\end{equation}


Cum sociis natoque penatibus et magnis dis parturient montes, nascetur ridiculus mus. Donec quam felis, ultricies nec, pellentesque eu, pretium quis, sem. Nulla consequat massa quis enim. Donec pede justo, fringilla vel, aliquet nec, vulputate eget, arcu. In enim justo, rhoncus ut, imperdiet a, venenatis vitae, justo. Nullam dictum felis eu pede mollis pretium. Integer tincidunt. Cras dapibus. Vivamus elementum semper nisi. Aenean vulputate eleifend tellus. Aenean leo ligula, porttitor eu, consequat vitae, eleifend ac, enim. Aliquam lorem ante, dapibus in, viverra quis, feugiat a, tellus. Phasellus viverra nulla ut metus varius laoreet. Quisque rutrum. Aenean imperdiet. Etiam ultricies nisi vel augue. Curabitur ullamcorper ultricies nisi. Nam eget dui. Etiam rhoncus. Maecenas tempus, tellus eget condimentum rhoncus, sem quam semper libero, sit amet adipiscing sem neque sed ipsum. Nam quam nunc, blandit vel, luctus pulvinar, hendrerit id, lorem. Maecenas nec odio et ante tincidunt tempus. Donec vitae sapien ut libero venenatis faucibus. Nullam quis ante. Etiam sit amet orci eget eros faucibus tincidunt. Duis leo. Sed fringilla mauris sit amet nibh. Donec sodales sagittis magna. Sed consequat, leo eget bibendum sodales, augue velit cursus nunc,

Lorem ipsum dolor sit amet, consectetuer adipiscing elit. Aenean commodo ligula eget dolor. Aenean massa. Cum sociis natoque penatibus et magnis dis parturient montes, nascetur ridiculus mus. Donec quam felis, ultricies nec, pellentesque eu, pretium quis, sem. Nulla consequat massa quis enim. Donec pede justo, fringilla vel, aliquet nec, vulputate eget, arcu. In enim justo, rhoncus ut, imperdiet a, venenatis vitae, justo. Nullam dictum felis eu pede mollis pretium. Integer tincidunt. Cras dapibus. Vivamus elementum semper nisi. Aenean vulputate eleifend tellus. Aenean leo ligula, porttitor eu, consequat vitae, eleifend ac, enim. Aliquam lorem ante, dapibus in, viverra quis, feugiat a, tellus. Phasellus viverra nulla ut metus varius laoreet. Quisque rutrum. Aenean imperdiet. Etiam ultricies nisi vel augue. Curabitur ullamcorper ultricies nisi. Nam eget dui. Etiam rhoncus. Maecenas tempus, tellus eget condimentum rhoncus, sem quam semper libero, sit amet adipiscing sem neque sed ipsum. Nam quam nunc, blandit vel, luctus pulvinar, hendrerit id, lorem. Maecenas nec odio et ante tincidunt tempus. Donec vitae sapien ut libero venenatis faucibus. Nullam quis ante. Etiam sit amet orci eget eros faucibus tincidunt. Duis leo. Sed fringilla mauris sit amet nibh. Donec sodales sagittis magna. Sed consequat, leo eget bibendum sodales, augue velit cursus nunc,


\chapter{El segundo cap\'{\i}tulo}\label{chap4}

Lorem ipsum dolor sit amet, consectetuer adipiscing elit. Aenean commodo ligula eget dolor. Aenean massa. Cum sociis natoque penatibus et magnis dis parturient montes, nascetur ridiculus mus. Donec quam felis, ultricies nec, pellentesque eu, pretium quis, sem. Nulla consequat massa quis enim. Donec pede justo, fringilla vel, aliquet nec, vulputate eget, arcu. In enim justo, rhoncus ut, imperdiet a, venenatis vitae, justo. Nullam dictum felis eu pede mollis pretium. Integer tincidunt. Cras dapibus. Vivamus elementum semper nisi. Aenean vulputate eleifend tellus. Aenean leo ligula, porttitor eu, consequat vitae, eleifend ac, enim. Aliquam lorem ante, dapibus in, viverra quis, feugiat a, tellus.
\begin{equation}\label{eq8}
e^{i\pi }+1=0.
\end{equation}
Phasellus viverra nulla ut metus varius laoreet. Quisque rutrum. Aenean imperdiet. Etiam ultricies nisi vel augue. Curabitur ullamcorper ultricies nisi. Nam eget dui. Etiam rhoncus. Maecenas tempus, tellus eget condimentum rhoncus, sem quam semper libero, sit amet adipiscing sem neque sed ipsum. Nam quam nunc, blandit vel, luctus pulvinar, hendrerit id, lorem. Maecenas nec odio et ante tincidunt tempus. Donec vitae sapien ut libero venenatis faucibus. Nullam quis ante. Etiam sit amet orci eget eros faucibus tincidunt. Duis leo. Sed fringilla mauris sit amet nibh. Donec sodales sagittis magna. Sed consequat, leo eget bibendum sodales, augue velit cursus nunc,
$$
e^{i\pi }+1=0.
$$

Lorem ipsum dolor sit amet, consectetuer adipiscing elit. Aenean commodo ligula eget dolor. Aenean massa. Cum sociis natoque penatibus et magnis dis parturient montes, nascetur ridiculus mus. Donec quam felis, ultricies nec, pellentesque eu, pretium quis, sem. Nulla consequat massa quis enim. Donec pede justo, fringilla vel, aliquet nec, vulputate eget, arcu. In enim justo, rhoncus ut, imperdiet a, venenatis vitae, justo. Nullam dictum felis eu pede mollis pretium.
\begin{equation}
e^{i\pi }+1=0.
\end{equation}


Integer tincidunt. Cras dapibus. Vivamus elementum semper nisi. Aenean vulputate eleifend tellus. Aenean leo ligula, porttitor eu, consequat vitae, eleifend ac, enim. Aliquam lorem ante, dapibus in, viverra quis, feugiat a, tellus. Phasellus viverra nulla ut metus varius laoreet. Quisque rutrum. Aenean imperdiet. Etiam ultricies nisi vel augue. Curabitur ullamcorper ultricies nisi.
\begin{equation}
e^{i\pi }+1=0.
\end{equation}
Nam eget dui. Etiam rhoncus. Maecenas tempus, tellus eget condimentum rhoncus, sem quam semper libero, sit amet adipiscing sem neque sed ipsum. Nam quam nunc, blandit vel, luctus pulvinar, hendrerit id, lorem. Maecenas nec odio et ante tincidunt tempus. Donec vitae sapien ut libero venenatis faucibus. Nullam quis ante. Etiam sit amet orci eget eros faucibus tincidunt. Duis leo. Sed fringilla mauris sit amet nibh. Donec sodales sagittis magna. Sed consequat, leo eget bibendum sodales, augue velit cursus nunc,
$$
e^{i\pi }+1=0.
$$

Lorem ipsum dolor sit amet, consectetuer adipiscing elit. Aenean commodo ligula eget dolor. Aenean massa. Cum sociis natoque penatibus et magnis dis parturient montes, nascetur ridiculus mus. Donec quam felis, ultricies nec, pellentesque eu, pretium quis, sem. Nulla consequat massa quis enim. Donec pede justo, fringilla vel, aliquet nec, vulputate eget, arcu. In enim justo, rhoncus ut, imperdiet a, venenatis vitae, justo. Nullam dictum felis eu pede mollis pretium. Integer tincidunt. Cras dapibus. Vivamus elementum semper nisi. Aenean vulputate eleifend tellus. Aenean leo ligula, porttitor eu, consequat vitae, eleifend ac, enim. Aliquam lorem ante, dapibus in, viverra quis, feugiat a, tellus. Phasellus viverra nulla ut metus varius laoreet. Quisque rutrum. Aenean imperdiet. Etiam ultricies nisi vel augue. Curabitur ullamcorper ultricies nisi. Nam eget dui. Etiam rhoncus. Maecenas tempus, tellus eget condimentum rhoncus, sem quam semper libero, sit amet adipiscing sem neque sed ipsum. Nam quam nunc, blandit vel, luctus pulvinar, hendrerit id, lorem. Maecenas nec odio et ante tincidunt tempus. Donec vitae sapien ut libero venenatis faucibus. Nullam quis ante. Etiam sit amet orci eget eros faucibus tincidunt. Duis leo. Sed fringilla mauris sit amet nibh. Donec sodales sagittis magna. Sed consequat, leo eget bibendum sodales, augue velit cursus nunc,

\section{Uno m\'as}

Lorem ipsum dolor sit amet, consectetuer adipiscing elit. Aenean commodo ligula eget dolor. Aenean massa. Cum sociis natoque penatibus et magnis dis parturient montes, nascetur ridiculus mus. Donec quam felis, ultricies nec, pellentesque eu, pretium quis, sem. Nulla consequat massa quis enim. Donec pede justo, fringilla vel, aliquet nec, vulputate eget, arcu. In enim justo, rhoncus ut, imperdiet a, venenatis vitae, justo. Nullam dictum felis eu pede mollis pretium. Integer tincidunt. Cras dapibus. Vivamus elementum semper nisi. Aenean vulputate eleifend tellus. Aenean leo ligula, porttitor eu, consequat vitae, eleifend ac, enim. Aliquam lorem ante, dapibus in, viverra quis, feugiat a, tellus. Phasellus viverra nulla ut metus varius laoreet. Quisque rutrum. Aenean imperdiet. Etiam ultricies nisi vel augue. Curabitur ullamcorper ultricies nisi. Nam eget dui. Etiam rhoncus. Maecenas tempus, tellus eget condimentum rhoncus, sem quam semper libero, sit amet adipiscing sem neque sed ipsum. Nam quam nunc, blandit vel, luctus pulvinar, hendrerit id, lorem. Maecenas nec odio et ante tincidunt tempus. Donec vitae sapien ut libero venenatis faucibus. Nullam quis ante. Etiam sit amet orci eget eros faucibus tincidunt. Duis leo. Sed fringilla mauris sit amet nibh. Donec sodales sagittis magna. Sed consequat, leo eget bibendum sodales, augue velit cursus nunc,
$$
e^{i\pi }+1=0.
$$

\section{Y otro}

Lorem ipsum dolor sit amet, consectetuer adipiscing elit. Aenean commodo ligula eget dolor. Aenean massa. Cum sociis natoque penatibus et magnis dis parturient montes, nascetur ridiculus mus. Donec quam felis, ultricies nec, pellentesque eu, pretium quis, sem. Nulla consequat massa quis enim. Donec pede justo, fringilla vel, aliquet nec, vulputate eget, arcu. In enim justo, rhoncus ut, imperdiet a, venenatis vitae, justo. Nullam dictum felis eu pede mollis pretium. Integer tincidunt. Cras dapibus. Vivamus elementum semper nisi. Aenean vulputate eleifend tellus. Aenean leo ligula, porttitor eu, consequat vitae, eleifend ac, enim. Aliquam lorem ante, dapibus in, viverra quis, feugiat a, tellus. Phasellus viverra nulla ut metus varius laoreet. Quisque rutrum. Aenean imperdiet. Etiam ultricies nisi vel augue. Curabitur ullamcorper ultricies nisi. Nam eget dui. Etiam rhoncus. Maecenas tempus, tellus eget condimentum rhoncus, sem quam semper libero, sit amet adipiscing sem neque sed ipsum. Nam quam nunc, blandit vel, luctus pulvinar, hendrerit id, lorem. Maecenas nec odio et ante tincidunt tempus. Donec vitae sapien ut libero venenatis faucibus. Nullam quis ante. Etiam sit amet orci eget eros faucibus tincidunt. Duis leo. Sed fringilla mauris sit amet nibh. Donec sodales sagittis magna. Sed consequat, leo eget bibendum sodales, augue velit cursus nunc,


Lorem ipsum dolor sit amet, consectetuer adipiscing elit. Aenean commodo ligula eget dolor. Aenean massa. Cum sociis natoque penatibus et magnis dis parturient montes, nascetur ridiculus mus. Donec quam felis, ultricies nec, pellentesque eu, pretium quis, sem. Nulla consequat massa quis enim. Donec pede justo, fringilla vel, aliquet nec, vulputate eget, arcu. In enim justo, rhoncus ut, imperdiet a, venenatis vitae, justo. Nullam dictum felis eu pede mollis pretium. Integer tincidunt. Cras dapibus. Vivamus elementum semper nisi. Aenean vulputate eleifend tellus. Aenean leo ligula, porttitor eu, consequat vitae, eleifend ac, enim. Aliquam lorem ante, dapibus in, viverra quis, feugiat a, tellus. Phasellus viverra nulla ut metus varius laoreet. Quisque rutrum. Aenean imperdiet. Etiam ultricies nisi vel augue. Curabitur ullamcorper ultricies nisi. Nam eget dui. Etiam rhoncus. Maecenas tempus, tellus eget condimentum rhoncus, sem quam semper libero, sit amet adipiscing sem neque sed ipsum. Nam quam nunc, blandit vel, luctus pulvinar, hendrerit id, lorem. Maecenas nec odio et ante tincidunt tempus. Donec vitae sapien ut libero venenatis faucibus. Nullam quis ante. Etiam sit amet orci eget eros faucibus tincidunt. Duis leo. Sed fringilla mauris sit amet nibh. Donec sodales sagittis magna. Sed consequat, leo eget bibendum sodales, augue velit cursus nunc,


Lorem ipsum dolor sit amet, consectetuer adipiscing elit. Aenean commodo ligula eget dolor. Aenean massa. Cum sociis natoque penatibus et magnis dis parturient montes, nascetur ridiculus mus. Donec quam felis, ultricies nec, pellentesque eu, pretium quis, sem. Nulla consequat massa quis enim. Donec pede justo, fringilla vel, aliquet nec, vulputate eget, arcu. In enim justo, rhoncus ut, imperdiet a, venenatis vitae, justo. Nullam dictum felis eu pede mollis pretium. Integer tincidunt. Cras dapibus. Vivamus elementum semper nisi. Aenean vulputate eleifend tellus. Aenean leo ligula, porttitor eu, consequat vitae, eleifend ac, enim. Aliquam lorem ante, dapibus in, viverra quis, feugiat a, tellus. Phasellus viverra nulla ut metus varius laoreet. Quisque rutrum. Aenean imperdiet. Etiam ultricies nisi vel augue. Curabitur ullamcorper ultricies nisi. Nam eget dui. Etiam rhoncus. Maecenas tempus, tellus eget condimentum rhoncus, sem quam semper libero, sit amet adipiscing sem neque sed ipsum. Nam quam nunc, blandit vel, luctus pulvinar, hendrerit id, lorem. Maecenas nec odio et ante tincidunt tempus. Donec vitae sapien ut libero venenatis faucibus. Nullam quis ante. Etiam sit amet orci eget eros faucibus tincidunt. Duis leo. Sed fringilla mauris sit amet nibh. Donec sodales sagittis magna. Sed consequat, leo eget bibendum sodales, augue velit cursus nunc,


Lorem ipsum dolor sit amet, consectetuer adipiscing elit. Aenean commodo ligula eget dolor. Aenean massa. Cum sociis natoque penatibus et magnis dis parturient montes, nascetur ridiculus mus. Donec quam felis, ultricies nec, pellentesque eu, pretium quis, sem. Nulla consequat massa quis enim. Donec pede justo, fringilla vel, aliquet nec, vulputate eget, arcu. In enim justo, rhoncus ut, imperdiet a, venenatis vitae, justo. Nullam dictum felis eu pede mollis pretium. Integer tincidunt. Cras dapibus. Vivamus elementum semper nisi. Aenean vulputate eleifend tellus. Aenean leo ligula, porttitor eu, consequat vitae, eleifend ac, enim. Aliquam lorem ante, dapibus in, viverra quis, feugiat a, tellus. Phasellus viverra nulla ut metus varius laoreet. Quisque rutrum. Aenean imperdiet. Etiam ultricies nisi vel augue. Curabitur ullamcorper ultricies nisi. Nam eget dui. Etiam rhoncus. Maecenas tempus, tellus eget condimentum rhoncus, sem quam semper libero, sit amet adipiscing sem neque sed ipsum. Nam quam nunc, blandit vel, luctus pulvinar, hendrerit id, lorem. Maecenas nec odio et ante tincidunt tempus. Donec vitae sapien ut libero venenatis faucibus. Nullam quis ante. Etiam sit amet orci eget eros faucibus tincidunt. Duis leo. Sed fringilla mauris sit amet nibh. Donec sodales sagittis magna. Sed consequat, leo eget bibendum sodales, augue velit cursus nunc,


Lorem ipsum dolor sit amet, consectetuer adipiscing elit. Aenean commodo ligula eget dolor. Aenean massa. Cum sociis natoque penatibus et magnis dis parturient montes, nascetur ridiculus mus. Donec quam felis, ultricies nec, pellentesque eu, pretium quis, sem. Nulla consequat massa quis enim. Donec pede justo, fringilla vel, aliquet nec, vulputate eget, arcu. In enim justo, rhoncus ut, imperdiet a, venenatis vitae, justo. Nullam dictum felis eu pede mollis pretium. Integer tincidunt. Cras dapibus. Vivamus elementum semper nisi. Aenean vulputate eleifend tellus. Aenean leo ligula, porttitor eu, consequat vitae, eleifend ac, enim. Aliquam lorem ante, dapibus in, viverra quis, feugiat a, tellus. Phasellus viverra nulla ut metus varius laoreet. Quisque rutrum. Aenean imperdiet. Etiam ultricies nisi vel augue. Curabitur ullamcorper ultricies nisi. Nam eget dui. Etiam rhoncus. Maecenas tempus, tellus eget condimentum rhoncus, sem quam semper libero, sit amet adipiscing sem neque sed ipsum. Nam quam nunc, blandit vel, luctus pulvinar, hendrerit id, lorem. Maecenas nec odio et ante tincidunt tempus. Donec vitae sapien ut libero venenatis faucibus. Nullam quis ante. Etiam sit amet orci eget eros faucibus tincidunt. Duis leo. Sed fringilla mauris sit amet nibh. Donec sodales sagittis magna. Sed consequat, leo eget bibendum sodales, augue velit cursus nunc,

\begin{thebibliography}{10}
\addcontentsline{toc}{chapter}{\bibname}

\bibitem{Abel} 
    \textsc{Abel, N.\,H.}: 
    Beweis eines Ausdrucks, von welchem die Binomial-Formel ein einzelner Fall ist. 
    \textit{J. Reine angew. Math.} {\bf1} (1826), 159--160.

\bibitem{S-W}
    \textsc{Stein, E.\,M. and Weiss, G.}
    \textit{Introduction to  Fourier analysis on Euclidean spaces.}
    Princeton Mathematical Series~32, Princeton University Press, Princeton, NJ, 1971.
    
    
\end{thebibliography}
\cleardoublepage


\end{document}
